\documentclass[12pt,a4paper]{article}
\usepackage[utf8]{inputenc}
\usepackage[brazil]{babel}
\usepackage{graphicx}
\usepackage{caption}
\usepackage{siunitx}
\usepackage{changepage}
\usepackage{setspace}
\usepackage{float}
\usepackage[a4paper]{geometry}
\usepackage{amsmath,amsfonts,amstext,amscd,bezier}
\usepackage{mathtools}
\usepackage{amsthm}
\usepackage{mathrsfs}
\usepackage{times}
\usepackage{indentfirst}
\usepackage{hyperref}
\usepackage{placeins}
\usepackage{array}
\usepackage{subfigure}
\usepackage{wrapfig}
\usepackage{setspace}
\usepackage{gensymb}
\usepackage{svg}
\usepackage[backend=bibtex,sorting=none]{biblatex}
\usepackage{verbatim}
\usepackage[autostyle]{csquotes}
\usepackage{minted}
\linespread{1.1}

\begin{document}
	\begin{titlepage}
		\begin{center}
			\large Universidade de São Paulo\\[0.2cm]
			\large Instituto de Física de São Carlos\\[7cm]
			\huge \textbf{Relatório 3 - IntroFisComp}\\[6cm]
			
			\large Alexandre de Taunay Voloch \\[0.2cm]
		\end{center}
	\end{titlepage}

\section{Tarefa 1}

Aqui o programa é bem simples, sendo apenas muito extenso por causa das diversas funções diferentes que precisamos programar. O programa imprime os valores corretos (analíticos) das derivadas e também os valores a serem colocados em cada tabela respectiva. Segue o programa:

\begin{minted}[
	mathescape,
	linenos,
	fontsize=\footnotesize,
	framesep=2mm,
	breaklines]
	{fortranfixed}
      implicit real*8 (a-h,o-z)

      real*8 valores_h(14)
      real*8 valores_tabela(7,14)
      data valores_h / 5d-1, 1d-1, 5d-2, 1d-2, 5.d-3, 1.d-3, 5.d-4,
     &1.d-4, 5.d-5, 1.d-5, 5.d-6, 1.d-6, 1.d-7, 1.d-8/

      x = 0.5d0

      do i=1,7
         do j=1,14
            valores_tabela(i,j) = 0.d0
         end do
      end do

      f1_correto = anal_f1(x)
      f2_correto = anal_f2(x)
      f3_correto = anal_f3(x)

      !write(*,*)f1_correto, f2_correto, f3_correto

      do i=1,14
         h = valores_h(i)
         valores_tabela(1,i) = f1_2f(x,h) - f1_correto
         valores_tabela(2,i) = f1_2t(x,h) - f1_correto
         valores_tabela(3,i) = f1_3s(x,h) - f1_correto
         valores_tabela(4,i) = f1_5s(x,h) - f1_correto
         valores_tabela(5,i) = f2_3s(x,h) - f2_correto
         valores_tabela(6,i) = f2_5s(x,h) - f2_correto
         valores_tabela(7,i) = f3_5a(x,h) - f3_correto
      end do

      ! fazer valores da tabela serem absolutos
      do j=1,7
         do i=1,14
            valores_tabela(j,i) = abs(valores_tabela(j,i))
         end do
      end do

      write(*,*)"Valores da tabela da primeira derivada:"
      ! printar valores da primeira tabela, de f'
      do i=1,14
         write(*,*) (valores_tabela(j,i), j=1,4)
      end do
      write(*,*)"Valor exato:", f1_correto

      !printar valores da segunda tabela, de f''
      write(*,*)"Valores da tabela da segunda derivada:"
      do i=1,14
         write(*,*) (valores_tabela(j,i), j=5,6)
      end do
      write(*,*)"Valor exato:", f2_correto

      write(*,*)"Valores da tabela da terceira derivada:"
      ! printar valores da terceira tabela, de f'''
      do i=1,14
         write(*,*) valores_tabela(7,i)
      end do
      write(*,*)"Valor exato:", f3_correto
      end

      function cf(x)
         real*8 x
         real*8 cf

         cf = dexp(x/2.d0)*dtan(2.d0*x)
      return
      end function cf

      function dsec2(x)
         ! sec²x
         real*8 x
         dsec2 = (1.d0/dcos(x))**2.d0
         return
      end function

      function anal_f1(x)
         ! f' analitico
         real*8 x
         real*8 anal_f1

         ds = dsec2(2.d0*x)
         dt = dtan(2.d0*x)

         anal_f1 = dexp(x/2.d0)*(2.d0*ds + 0.5d0*dt)
      return
      end function

      function anal_f2(x)
         ! f'' analitico
         real*8 x
         real*8 anal_f2

         ds = dsec2(2.d0*x)
         dt = dtan(2.d0*x)

         anal_f2 = dexp(x/2.d0)*(2.d0*ds + 0.25d0*dt + 8.d0*ds*dt)
      end function

      function anal_f3(x)
         ! f'' analitico
         real*8 x
         real*8 anal_f3

         ds = dsec2(2.d0*x)
         dt = dtan(2.d0*x)

         anal_f3 = dexp(x/2.d0)*(1.5d0*ds + (1.d0/8.d0)*dt + 12.d0*ds*dt
     &   + 16.d0*(ds**2.d0) + 32.d0*ds*(dt**2.d0))
      end function

      function f1_2f(x,h)
         real*8 cf
         external cf
         ! f'_{2f}
         real*8 x,h,f1_2f

         f1_2f = (cf(x+h)-cf(x))/h

      end function

      function f1_2t(x,h)
         ! f'_{2t}
         real*8 cf
         external cf
         real*8 x,h,f1_2t
         f1_2t = (cf(x)-cf(x-h))/h

      end function

      function f1_3s(x,h)
         real*8 cf
         external cf
         real*8 x,h,f1_3s
         f1_3s = (cf(x+h)-cf(x-h))/(2.d0*h)

      end function

      function f1_5s(x,h)
         real*8 cf
         external cf
         real*8 x,h, f1_5s
         f1_5s = (cf(x-2.d0*h) - 8.d0*cf(x-h)
     &   + 8.d0*cf(x+h) - cf(x+2.d0*h))/(12.d0 * h)
      end function

      function f2_3s(x,h)
         real*8 cf
         external cf
         real*8 x,h, f2_3s
         f2_3s = (cf(x+h) -2.d0*cf(x) + cf(x-h))/(h**2.d0)
      end function

      function f2_5s(x,h)
         real*8 cf
         external cf
         real*8 x,h, f2_5s
         f2_5s = (-1.d0*cf(x-2.d0*h) + 16.d0*cf(x-h) -30.d0*cf(x)
     &   +16.d0*cf(x+h) - cf(x+2.d0*h)) / (12.d0*(h**2.d0))
      end function

      function f3_5a(x,h)
         real*8 cf
         external cf
         real*8 x,h, f3_5a
         f3_5a = (-1.d0*cf(x- 2.d0*h) + 2.d0*cf(x-h) - 2.d0*cf(x+h)
     &   + cf(x+2.d0*h))/(2.d0*(h**3.d0))
      end function
\end{minted}

Executando-o e tabelando os resultados, obtemos as seguintes tabelas:

\begin{table}[H]
\footnotesize
\centering
\begin{adjustwidth}{-2cm}{2cm}
\begin{tabular}{|c|c|c|c|c|}
\hline
$h$ & $f'_{2f}$ & $f'_{2t}$ & $f'_{3s}$ & $f'_{5s}$ \\ \hline
$5 \times 10^{-1}$ & $2.10013274476838 \times 10^{1}$ & $5.79727963749901$ & $1.33993035425914 \times 10^{1}$ & $1.47520006627981 \times 10^{1}$ \\ \hline
$1 \times 10^{-1}$ & $4.92612234674952$ & $2.37530455157099$ & $1.27540889758926$ & $1.22780673941021$ \\ \hline
$5 \times 10^{-2}$ & $1.94152325544609$ & $1.36425502733287$ & $2.88634114056608 \times 10^{-1}$ & $4.02908137876121 \times 10^{-2}$ \\ \hline
$1 \times 10^{-2}$ & $3.32088970649423 \times 10^{-1}$ & $3.09677366242413 \times 10^{-1}$ & $1.12058022035040 \times 10^{-2}$ & $5.49830810037122 \times 10^{-5}$ \\ \hline
$5 \times 10^{-3}$ & $1.63093714957750 \times 10^{-1}$ & $1.57495704052170 \times 10^{-1}$ & $2.79900545278977 \times 10^{-3}$ & $3.26013078932874 \times 10^{-6}$ \\ \hline
$1 \times 10^{-3}$ & $3.21616449508326 \times 10^{-2}$ & $3.19374617383090 \times 10^{-2}$ & $1.12091606261799 \times 10^{-4}$ & $1.65651719896687 \times 10^{-7}$ \\ \hline
$5 \times 10^{-4}$ & $1.60527810142845 \times 10^{-2}$ & $1.59964790257945 \times 10^{-2}$ & $2.81509942450242 \times 10^{-5}$ & $1.70790166009738 \times 10^{-7}$ \\ \hline
$1 \times 10^{-4}$ & $3.20620694131790 \times 10^{-3}$ & $3.20362629620163 \times 10^{-3}$ & $1.29032255813399 \times 10^{-6}$ & $1.71131150139558 \times 10^{-7}$ \\ \hline
$5 \times 10^{-5}$ & $1.60290909571970 \times 10^{-3}$ & $1.60200723216164 \times 10^{-3}$ & $4.50931779027997 \times 10^{-7}$ & $1.71134852067212 \times 10^{-7}$ \\ \hline
$1 \times 10^{-5}$ & $3.20673902141522 \times 10^{-4}$ & $3.20309297709542 \times 10^{-4}$ & $1.82302215989694 \times 10^{-7}$ & $1.71100063894869 \times 10^{-7}$ \\ \hline
$5 \times 10^{-6}$ & $1.60419847670568 \times 10^{-4}$ & $1.60071918788418 \times 10^{-4}$ & $1.73964441074759 \times 10^{-7}$ & $1.71177779506593 \times 10^{-7}$ \\ \hline
$1 \times 10^{-6}$ & $3.22203966156565 \times 10^{-5}$ & $3.18781076664720 \times 10^{-5}$ & $1.71144474592211 \times 10^{-7}$ & $1.71088963440980 \times 10^{-7}$ \\ \hline
$1 \times 10^{-7}$ & $3.37147336537669 \times 10^{-6}$ & $3.03451348671047 \times 10^{-6}$ & $1.68479939333110 \times 10^{-7}$ & $1.66814604796173 \times 10^{-7}$ \\ \hline
$1 \times 10^{-8}$ & $5.64829559124291 \times 10^{-7}$ & $1.90122097620815 \times 10^{-7}$ & $1.87353730751738 \times 10^{-7}$ & $1.91054473575036 \times 10^{-7}$ \\ \hline
Exato & \multicolumn{4}{c|}{$9.79678184270445$} \\ \hline
\end{tabular}
\caption{Erros absolutos para a primeira derivada}
\end{adjustwidth}
\end{table}

Como podemos ver, para a primeira derivada um maior valor de $h$ equivale, em geral, a uma maior precisão. No caso das últimas duas técnicas ($f'_{3s}$ e $f'_{5s}$) a precisão é maximizada com $h = 10^{-7}$ e depois torna a diminuir levemente.


\begin{table}[H]
\centering
\begin{tabular}{|c|c|c|}
\hline
$h$ & $f''_{3s}$ & $f''_{5s}$ \\ \hline
$5 \times 10^{-1}$ & $9.45064193068224 \times 10^{1}$ & $1.02804389820175 \times 10^{2}$ \\ \hline
$1 \times 10^{-1}$ & $8.91594529675227$ & $8.60379834714472$ \\ \hline
$5 \times 10^{-2}$ & $2.01724196912639$ & $2.82325806748908 \times 10^{-1}$ \\ \hline
$1 \times 10^{-2}$ & $7.83100027307881 \times 10^{-2}$ & $3.85608147027483 \times 10^{-4}$ \\ \hline
$5 \times 10^{-3}$ & $1.95601155312346 \times 10^{-2}$ & $2.31802004719839 \times 10^{-5}$ \\ \hline
$1 \times 10^{-3}$ & $7.83002688862666 \times 10^{-4}$ & $8.24995879611379 \times 10^{-7}$ \\ \hline
$5 \times 10^{-4}$ & $1.96393705252262 \times 10^{-4}$ & $8.58117530810887 \times 10^{-7}$ \\ \hline
$1 \times 10^{-4}$ & $8.68874256809704 \times 10^{-6}$ & $8.72772460525084 \times 10^{-7}$ \\ \hline
$5 \times 10^{-5}$ & $2.87117391906122 \times 10^{-6}$ & $9.91196259292337 \times 10^{-7}$ \\ \hline
$1 \times 10^{-5}$ & $3.70134640093056 \times 10^{-6}$ & $4.44149507927705 \times 10^{-6}$ \\ \hline
$5 \times 10^{-6}$ & $2.96053443378241 \times 10^{-5}$ & $4.81090614243840 \times 10^{-5}$ \\ \hline
$1 \times 10^{-6}$ & $1.80595675701056 \times 10^{-4}$ & $4.95158866002043 \times 10^{-4}$ \\ \hline
$1 \times 10^{-7}$ & $3.84551655812402 \times 10^{-2}$ & $4.40062807043802 \times 10^{-2}$ \\ \hline
$1 \times 10^{-8}$ & $1.13968419880578 \times 10^{1}$ & $1.54676597450168 \times 10^{1}$ \\ \hline
Exato & \multicolumn{2}{c|}{$64.0983236864528$} \\ \hline
\end{tabular}
\caption{Erros absolutos para a segunda derivada}
\end{table}

Para a segunda derivada, podemos ver que no caso de $f''_{3s}$, a maior precisão é atingida em $h = 5\cdot 10^{-5}$, enquanto que para $f''_{5s}$, a maior precisão é atingida muito mais rapidamente com $h = 1\cdot10^{-3}$.

\begin{table}[H]
\centering
\begin{tabular}{|c|c|}
\hline
$h$ & $f'''_{5a}$ \\ \hline
$5 \times 10^{-1}$ & $6.39049869974093 \times 10^{2}$ \\ \hline
$1 \times 10^{-1}$ & $8.30414781340628 \times 10^{2}$ \\ \hline
$5 \times 10^{-2}$ & $1.17905225967068 \times 10^{2}$ \\ \hline
$1 \times 10^{-2}$ & $4.13251621152244$ \\ \hline
$5 \times 10^{-3}$ & $1.02913919815319$ \\ \hline
$1 \times 10^{-3}$ & $4.11268346746283 \times 10^{-2}$ \\ \hline
$5 \times 10^{-4}$ & $1.02952751469729 \times 10^{-2}$ \\ \hline
$1 \times 10^{-4}$ & $6.86961002543285 \times 10^{-4}$ \\ \hline
$5 \times 10^{-5}$ & $1.97757425655709 \times 10^{-3}$ \\ \hline
$1 \times 10^{-5}$ & $7.25440551477732 \times 10^{-1}$ \\ \hline
$5 \times 10^{-6}$ & $4.49260766426050$ \\ \hline
$1 \times 10^{-6}$ & $5.38078608396029$ \\ \hline
$1 \times 10^{-7}$ & $5.54439997711719 \times 10^{5}$ \\ \hline
$1 \times 10^{-8}$ & $4.44089881364663 \times 10^{8}$ \\ \hline
Exato & $671.514600859054$ \\ \hline
\end{tabular}
\caption{Erros absolutos para a terceira derivada}
\end{table}

Aqui na terceira derivada, vemos que a precisão é menor, e que atinge seu máximo em $h = 10^{-4}$, depois passando a diminuir muito e praticamente divergir para $h$ muito pequeno.

\section{Tarefa 2}

Aqui o programa imprime primeiro o valor analítico da integral, que vale $\frac{e-1}{4e\pi^2}$ (fonte: Wolfram), e depois imprime as diferenças observadas utilizando cada método de aproximação da integral. O programa é o seguinte:

\begin{minted}[
	mathescape,
	linenos,
	fontsize=\footnotesize,
	framesep=2mm,
	breaklines]
	{fortranfixed}
      implicit real*8 (a-h,o-z)

      pi = 4.d0*datan(1.d0)
      e = dexp(1.d0)

      valor_analitico = (e - 1.d0)/(e + (4.d0*e*(pi**2.d0)))

      write(*,*)valor_analitico

      do i=2,12
         h = 1.d0/(3.d0*(2.d0**i))
         xt = trapezio(h)
         xs = simpson(h)
         xb = boole(h)

         et = abs(valor_analitico-xt)
         es = abs(valor_analitico-xs)
         eb = abs(valor_analitico-xb)

         write(*,*)et, es, eb
      end do

      end

      function cf(x)
         real*8 x
         real*8 cf
         pi = 4.d0*datan(1.d0)
         cf = dexp(-x)*dcos(2.d0*pi*x)
      end function

      function trapezio(h)
         real*8 h, cf, trapezio, a, b, soma
         integer*16 nparticoes
         external cf

         a = 0.d0
         b = 1.d0
         nparticoes = dint((b-a)/h)

         soma = 0.5d0 * (cf(a) + cf(b))
         do i=1,(nparticoes-1)
            soma = soma + cf(a + dble(i)*h)
         end do

         trapezio = h * soma
      end function

      function simpson(h)
         real*8 h, cf, simpson, a, b, soma
         integer*16 i, nparticoes
         external cf

         a = 0.d0
         b = 1.d0
         nparticoes = dint((b-a)/h)

         soma = cf(a) + cf(b)

         do i=1,nparticoes-1,2
            soma = soma + 4.d0*cf(a + dble(i)*h)
         end do

         do i=2,nparticoes-1,2
            soma = soma + 2.d0*cf(a + dble(i)*h)
         end do

         simpson = (h/3.d0) * soma
      end function

      function boole(h)
         real*8 h, cf, boole, a, b, soma
         integer*16 nparticoes, i
         external cf

         a = 0.d0
         b = 1.d0
         nparticoes = dint((b-a)/h)

         soma = 0.d0
         do i=0,nparticoes-1,4
            soma = soma + (2.d0*h/45.d0) * (7.d0*cf(a + dble(i)*h) +
     &          32.d0*cf(a + dble(i+1)*h) + 12.d0*cf(a + dble(i+2)*h) +
     &          32.d0*cf(a + dble(i+3)*h) + 7.d0*cf(a + dble(i+4)*h))
         end do

         boole = soma
      end function

\end{minted}

Tabelando esses resultados, temos

\begin{table}[H]
\centering
\begin{tabular}{|c|c|c|c|}
\hline
\( h^{-1} \) & Regra do Trapézio & Regra de Simpson & Regra de Boole \\
\hline
\hline
\( 3 \times 2^2 \) & 3.7084\( \times 10^{-4} \) & 2.0955\( \times 10^{-5} \) & 4.0327\( \times 10^{-6} \) \\
\( 3 \times 2^3 \) & 9.1773\( \times 10^{-5} \) & 1.2502\( \times 10^{-6} \) & 6.3421\( \times 10^{-8} \) \\
\( 3 \times 2^4 \) & 2.2892\( \times 10^{-5} \) & 6.8806\( \times 10^{-8} \) & 9.9554\( \times 10^{-9} \) \\
\( 3 \times 2^5 \) & 5.7261\( \times 10^{-6} \) & 4.2771\( \times 10^{-9} \) & 9.1493\( \times 10^{-9} \) \\
\( 3 \times 2^6 \) & 1.4382\( \times 10^{-6} \) & 8.8331\( \times 10^{-9} \) & 9.1368\( \times 10^{-9} \) \\
\( 3 \times 2^7 \) & 3.6638\( \times 10^{-7} \) & 9.1177\( \times 10^{-9} \) & 9.1366\( \times 10^{-9} \) \\
\( 3 \times 2^8 \) & 9.8446\( \times 10^{-8} \) & 9.1354\( \times 10^{-9} \) & 9.1366\( \times 10^{-9} \) \\
\( 3 \times 2^9 \) & 3.1464\( \times 10^{-8} \) & 9.1365\( \times 10^{-9} \) & 9.1366\( \times 10^{-9} \) \\
\( 3 \times 2^{10} \) & 1.4718\( \times 10^{-8} \) & 9.1366\( \times 10^{-9} \) & 9.1366\( \times 10^{-9} \) \\
\( 3 \times 2^{11} \) & 1.0532\( \times 10^{-8} \) & 9.1366\( \times 10^{-9} \) & 9.1366\( \times 10^{-9} \) \\
\( 3 \times 2^{12} \) & 9.4855\( \times 10^{-9} \) & 9.1366\( \times 10^{-9} \) & 9.1366\( \times 10^{-9} \) \\
\hline
\hline
Exato & \multicolumn{3}{|c|}{$1.5616236904490828 \times 10^{-2}$} \\
\hline
\end{tabular}
\caption{Diferenças entre os métodos e o valor analítico}
\end{table}

Aqui podemos ver que a regra do trapézio vai aumentando de precisão conforme $h$ diminui, e não chegamos num limite de precisão. Portanto, para este método o valor de $h$ ótimo depende da precisão desejada. Já a regra de Simpson chega rapidamente numa precisão boa, com $h^{-1} = 3\cdot2^{6}$ e praticamente não aumenta a partir dali. Já para a regra de Boole, chegamos numa precisão boa muito rapidamente, com $h^{-1} = 3\cdot2^{4}$ e também não aumenta quase nada a partir dali. Portanto, estas últimas duas técnicas permitem utilizar um valor maior de $h$, que significa menos cálculos realizados no total e, portanto, maior rapidez na execução do programa (lembrando que $h$ representa as subdivisões da "malha" do eixo x no qual estamos integrando).

\section{Tarefa 3}

Aqui utilizamos os métodos de acordo com a forma que são explicados no projeto. Primeiro o código procura os pontos onde há mudança de sinal (o início do método de bisseção) e depois utiliza estes como ponto de partida para cada um dos métodos de busca de raízes. Os métodos são executados até que atinjam a tolerância desejada, que neste caso é de $\epsilon = 10^{-6}$. O programa é o seguinte:

\begin{minted}[
	mathescape,
	linenos,
	fontsize=\footnotesize,
	framesep=2mm,
	breaklines]
	{fortranfixed}
      implicit real*8 (a-h,o-z)

      real*8 espac,busca_direta, xm, tolerancia, dif
      real*8 a(3), b(3)
      real*8 xnewt(3), asec(3), bsec(3), difs(3)
      real*8 resultados(3,3,100)  ! tabela de resultados
      logical terminamos

      !espac = 1d-1*dsqrt(5.d0)
      espac = 0.1d0
      data a / -10d0, -10d0, -10d0 /
      data b / -10d0, -10d0, -10d0 /

      tolerancia = 1d-6

      do i=1,3
         dif = 1.d0
         do while(dif.gt.0d0)
            a(i) = a(i) + espac
            b(i) = a(i) + espac
            dif = cf(b(i))*cf(a(i))
      !      write(1,*)a(i),b(i),dif
         end do
         write(*,*)"encontramos ponto de bisseção:",a(i),b(i)
         if(i.lt.3) then
            a(i+1)=b(i)
            b(i+1)=b(i)
         end if
      end do

      write(*,*)"Terminamos de pesquisar"
      xnewt = a
      asec = a
      bsec =  b

      ! calcular usando o método da bisseção

      write(*,*)"Método da bisseção:"

      do i=1,3
      write(*,*)"tentando i=", i
         niter = 0
         dif = 1.d0
         do while(dif.ge.tolerancia)
            xm = (b(i) + a(i))/2d0
            if( (cf(b(i))*cf(xm)).gt.0d0) then
               b(i) = xm
            else
               a(i) = xm
            end if

            dif = abs(b(i)-a(i))
            if (dif.lt.tolerancia) then
               !calcular xm novamente para imprimir o valor correto
               xm = (b(i) + a(i))/2d0
            end if
            write(*,*)xm
         end do
      end do

      write(*,*)"Método de Newton:"
      do i=1,3
      write(*,*)"tentando i=", i
         niter = 0
         dif = 1.d0
         do while(dif.ge.tolerancia)
            x_newt_antigo = xnewt(i) ! variável temporária p/ calcular precisão
            xnewt(i) = xnewt(i) - cf(xnewt(i))/df(xnewt(i))

            dif = abs(xnewt(i) - x_newt_antigo)
            write(*,*)xnewt(i)
         end do
      end do

      write(*,*)"Método da Secante:"
      do i=1,3
      write(*,*)"tentando i=", i
         niter = 0
         dif = 1.d0
         do while(dif.ge.tolerancia)
            xtemp = bsec(i) ! variável temporária pra guardar o valor antigo de x_n

            bsec(i) = bsec(i) - (cf(bsec(i)) * (bsec(i)-asec(i))
     &      / (cf(bsec(i))-cf(asec(i))))

            asec(i) = xtemp

            dif = abs(bsec(i)-asec(i))
            write(*,*)bsec(i)
         end do
      end do

      end

      function cf(x)
         real*8 cf,x
         cf = x**3d0 - 4d0*(x**2d0) - 59d0*x + 126d0
      end function

      function df(x)
         real*8 df,x
         df = 3d0*(x**2d0) - 8d0*x - 59d0
      end function

\end{minted}

Executando-o, para o intervalo de espaçamento requisitado de $0.1$, obtemos os seguintes resultados:

\begin{table}[H]
\centering
\begin{tabular}{|c|c|c|c|}
\hline
iteração & $r_1$ & $r_2$ & $r_3$ \\
\hline
0 & entre -7.01 e -6.9 & entre 1.99999 e 2.1 & entre 8.999 e 9.1 \\
\hline
1 & -6.9500000000000108 & 2.0499999999999821 & 9.0499999999999652 \\
\hline
2 & -6.9750000000000103 & 2.0249999999999821 & 9.0249999999999666 \\
\hline
3 & -6.9875000000000105 & 2.0124999999999820 & 9.0124999999999673 \\
\hline
4 & -6.9937500000000110 & 2.0062499999999819 & 9.0062499999999659 \\
\hline
5 & -6.9968750000000108 & 2.0031249999999821 & 9.0031249999999652 \\
\hline
6 & -6.9984375000000103 & 2.0015624999999821 & 9.0015624999999666 \\
\hline
7 & -6.9992187500000105 & 2.0007812499999820 & 9.0007812499999673 \\
\hline
8 & -6.9996093750000110 & 2.0003906249999819 & 9.0003906249999659 \\
\hline
9 & -6.9998046875000108 & 2.0001953124999821 & 9.0001953124999652 \\
\hline
10 & -6.9999023437500103 & 2.0000976562499821 & 9.0000976562499666 \\
\hline
11 & -6.9999511718750105 & 2.0000488281249820 & 9.0000488281249673 \\
\hline
12 & -6.9999755859375110 & 2.0000244140624819 & 9.0000244140624659 \\
\hline
13 & -6.9999877929687608 & 2.0000122070312321 & 9.0000122070312152 \\
\hline
14 & -6.9999938964843853 & 2.0000061035156071 & 9.0000061035155916 \\
\hline
15 & -6.9999969482421980 & 2.0000030517577945 & 9.0000030517577798 \\
\hline
16 & -6.9999984741211048 & 2.0000015258788881 & 9.0000015258788721 \\
\hline
17 & -6.9999996185302837 & 2.0000003814697087 & 9.0000003814696932 \\
\hline
Exato & -7.0000000000000000 & 2.0000000000000000 & 9.0000000000000000 \\
\hline
\end{tabular}
\caption{Método da Bisseção}
\end{table}

\begin{table}[H]
\centering
\begin{tabular}{|c|c|c|c|}
\hline
iteração & $r_1$ & $r_2$ & $r_3$ \\
\hline
0 & entre -7.01 e -6.9 & entre 1.99999 e 2.1 & entre 8.999 e 9.1 \\
\hline
1 & -7.0000000000000000 & 2.0000000000000000 & 9.0000000000000000 \\
\hline
Exato & -7.0000000000000000 & 2.0000000000000000 & 9.0000000000000000 \\
\hline
\end{tabular}
\caption{Método de Newton}
\end{table}

\begin{table}[H]
\centering
\begin{tabular}{|c|c|c|c|}
\hline
iteração & $r_1$ & $r_2$ & $r_3$ \\
\hline
0 & entre -7.01 e -6.9 & entre 1.99999 e 2.1 & entre 8.999 e 9.1 \\
\hline
1 & -7.0000000000000000 & 2.0000000000000000 & 8.9999999999999982 \\
\hline
2 & -7.0000000000000000 & 2.0000000000000000 & 9.0000000000000000 \\
\hline
Exato & -7.0000000000000000 & 2.0000000000000000 & 9.0000000000000000 \\
\hline
\end{tabular}
\caption{Método da Secante}
\end{table}

Podemos variar um pouco o valor do espaçamento para dar um pouco mais de trabalho para os métodos de Newton e Secante. No caso, mudando o intervalo para $\frac{1}{10}\sqrt{5} \approx 0.223$, temos

\begin{table}[H]
\centering
\begin{tabular}{|c|c|c|c|}
\hline
iteração & \( r_1 \) & \( r_2 \) & \( r_3 \) \\
\hline
0 & entre -7.09 e 6.86 & entre 1.85 e 2.07 & entre 8.78 e 9.007 \\
\hline
1 & -6.9813082303752889 & 1.9629636796238739 & 8.8947744098732251 \\
\hline
2 & -7.0372099298127839 & 2.0188653790613689 & 8.9506761093107201 \\
\hline
3 & -7.0092590800940364 & 1.9909145293426214 & 8.9786269590294658 \\
\hline
4 & -6.9952836552346627 & 2.0048899542019951 & 8.9926023838888405 \\
\hline
5 & -7.0022713676643491 & 1.9979022417723082 & 8.9995900963185278 \\
\hline
6 & -6.9987775114495054 & 2.0013960979871515 & 9.0030839525333697 \\
\hline
7 & -7.0005244395569273 & 1.9996491698797298 & 9.0013370244259487 \\
\hline
8 & -6.9996509755032168 & 2.0005226339334405 & 9.0004635603722392 \\
\hline
9 & -7.0000877075300725 & 2.0000859019065853 & 9.0000268283453835 \\
\hline
10 & -6.9998693415166446 & 1.9998675358931575 & 8.9998084623319556 \\
\hline
11 & -6.9999785245233586 & 1.9999767188998714 & 8.9999176453386696 \\
\hline
12 & -7.0000331160267155 & 2.0000313104032283 & 8.9999722368420265 \\
\hline
13 & -7.0000058202750370 & 2.0000040146515499 & 8.9999995325937050 \\
\hline
14 & -6.9999921723991978 & 1.9999903667757106 & 9.0000131804695442 \\
\hline
15 & -6.9999989963371174 & 1.9999971907136302 & 9.0000063565316246 \\
\hline
16 & -7.0000024083060772 & 2.0000006026825901 & 9.0000029445626648 \\
\hline
17 & -7.0000007023215973 & 1.9999988966981102 & 9.0000012385781858 \\
\hline
18 & -7.0000002758254780 & 2.0000001761864699 & 8.9999999590898252 \\
\hline
Exato & -7.0000000000000000 & 2.0000000000000000 & 9.0000000000000000 \\
\hline
\end{tabular}
\caption{Método da Bisseção com intervalo modificado}
\end{table}

\begin{table}[H]
\centering
\begin{tabular}{|c|c|c|c|}
\hline
iteração & \( r_1 \) & \( r_2 \) & \( r_3 \) \\
\hline
0 & entre -7.09 e 6.86 & entre 1.85 e 2.07 & entre 8.78 e 9.007 \\
\hline
1 & -7.0014686344168542 & 1.9994063811274549 & 9.0104043919142427 \\
\hline
2 & -7.0000003743126573 & 1.9999999888202884 & 9.0000221555644782 \\
\hline
3 & -7.0000000000000240 & 2.0000000000000000 & 9.0000000001008029 \\
\hline
4 & - & - & 9.0000000000000000 \\
\hline
Exato & -7.0000000000000000 & 2.0000000000000000 & 9.0000000000000000 \\
\hline
\end{tabular}
\caption{Método de Newton com intervalo modificado}
\end{table}

\begin{table}[H]
\centering
\begin{tabular}{|c|c|c|c|}
\hline
iteração & \( r_1 \) & \( r_2 \) & \( r_3 \) \\
\hline
0 & entre -7.09 e 6.86 & entre 1.85 e 2.07 & entre 8.78 e 9.007 \\
\hline
1 & -6.9978801131115222 & 2.0003394867587136 & 8.9996965282300696 \\
\hline
2 & -7.0000488919289863 & 1.9999991618752768 & 8.9999995904863042 \\
\hline
3 & -6.9999999820010332 & 2.0000000000090346 & 9.0000000000255227 \\
\hline
4 & -6.9999999999998472 & - & 9.0000000000000000 \\
\hline
Exato & -7.0000000000000000 & 2.0000000000000000 & 9.0000000000000000 \\
\hline
\end{tabular}
\caption{Método da Secante com intervalo modificado}
\end{table}

Em suma, podemos ver que mesmo com um intervalo com valor estranho os métodos de Newton-Raphson e da Secante convergem muito rapidamente ao valor exato da raiz, enquanto que o método da bisseção converge muito mais lentamente.

\end{document}